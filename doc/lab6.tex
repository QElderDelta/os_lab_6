\documentclass[a4paper, 12pt]{article}
\usepackage{cmap}
\usepackage[12pt]{extsizes}			
\usepackage{mathtext} 				
\usepackage[T2A]{fontenc}			
\usepackage[utf8]{inputenc}			
\usepackage[english,russian]{babel}
\usepackage{setspace}
\singlespacing
\usepackage{amsmath,amsfonts,amssymb,amsthm,mathtools}
\usepackage{fancyhdr}
\usepackage{soulutf8}
\usepackage{euscript}
\usepackage{mathrsfs}
\usepackage{listings}
\pagestyle{fancy}
\usepackage{indentfirst}
\usepackage[top=10mm]{geometry}
\rhead{}
\lhead{}
\renewcommand{\headrulewidth}{0mm}
\usepackage{tocloft}
\renewcommand{\cftsecleader}{\cftdotfill{\cftdotsep}}
\usepackage[dvipsnames]{xcolor}

\lstdefinestyle{mystyle}{ 
	keywordstyle=\color{OliveGreen},
	numberstyle=\tiny\color{Gray},
	stringstyle=\color{BurntOrange},
	basicstyle=\ttfamily\footnotesize,
	breakatwhitespace=false,         
	breaklines=true,                 
	captionpos=b,                    
	keepspaces=true,                 
	numbers=left,                    
	numbersep=5pt,                  
	showspaces=false,                
	showstringspaces=false,
	showtabs=false,                  
	tabsize=2
}

\lstset{style=mystyle}

\begin{document}
\thispagestyle{empty}	
\begin{center}
	Московский авиационный институт
	
	(Национальный исследовательский университет)
	
	Факультет "Информационные технологии и прикладная математика"
	
	Кафедра "Вычислительная математика и программирование"
	
\end{center}
\vspace{40ex}
\begin{center}
	\textbf{\large{Лабораторная работа №6 по курсу\linebreak \textquotedblleft Операционные системы\textquotedblright}}
\end{center}
\vspace{35ex}
\begin{flushright}
	\textit{Студент: } Живалев Е.А.
	
	\vspace{2ex}
	\textit{Группа: } М8О-206Б
	
	\vspace{2ex}
	\textit{Преподаватель: } Соколов А.А.
	
	\vspace{2ex}
	\textit{Вариант: } 26
	
	\vspace{2ex}
	\textit{Оценка: } \underline{\quad\quad\quad\quad\quad\quad}
	
	 \vspace{2ex}
	\textit{Дата: } \underline{\quad\quad\quad\quad\quad\quad}
	
	\vspace{2ex}
	\textit{Подпись: } \underline{\quad\quad\quad\quad\quad\quad}
	
\end{flushright}

\vspace{5ex}

\begin{vfill}
	\begin{center}
		Москва, 2019
	\end{center}	
\end{vfill}
\newpage

\section{Задание}

Реализовать распределенную систему по асинхронной обработке запросов. В данной
распределенной системе должно существовать 2 вида узлов: «управляющий» и
«вычислительный». Необходимо объединить данные узлы в соответствии с той топологией,
которая определена вариантом. Связь между узлами необходимо осуществить при помощи
технологии очередей сообщений. Также в данной системе необходимо предусмотреть проверку
доступности узлов в соответствии с вариантом.

Тип топологии: Дерево общего вида
Тип команды: Локальный таймер
Тип проверки доступности узлов: Пинг узла с указанным id

\section{Описание работы программы}

Программа разбита на файлы socketRoutine.hpp, socketRoutine.cpp(отвечают за работу с сокетами - получение и отправка сообщений, создание сокета), calcNode.cpp(содержат описание логики вычислительного узла), handlerNode.cpp\linebreak(содержит описание логики управляющего узла).

Каждый вычислительный узел при создании получает номер порта родителя, к которому он должен подключиться, а также свой id. Внутри себя он содержит 3 хэш-таблицы, содержащие сокеты детей узла, их идентификаторы процессов и номера портов.

При получении нового сообщения, адресованного конкретному узлу, строится путь до этого узла - вектор, содержащий id узлов на пути от управляющего к указанному и сообщение посылается в основной сокет, откуда согласно полученному пути пересылается через другие сокеты к необходимому узлу.
\newpage

\section{Исходный код}
\textbf{\large{socketRoutine.hpp}}
\lstinputlisting[language=C]{../socketRoutine.hpp}

\textbf{\large{socketRoutine.cpp}}
\lstinputlisting[language=C]{../socketRoutine.cpp}

\textbf{\large{calcNode.cpp}}
\lstinputlisting[language=C]{../calcNode.cpp}

\textbf{\large{handlerNode.cpp}}
\lstinputlisting[language=C]{../handlerNode.cpp}
\newpage
\section{Консоль}

\begin{verbatim}
qelderdelta@qelderdelta-UX331UA:~/Study/OS/lab6/build$ ./main
create 2 3
Ok: 13276
create 3 2
Ok: 13279
create 4 2
Ok: 13282
create 5 3
Ok: 13285
ping 5
Ok: 1
ping 6
Error: Not found
ping 2
Ok: 1
exec 4 time
Ok: 4: 0
exec 4 start
Ok:4
exec 4 stop
Ok:4
exec 4 time
Ok: 4: 10909
exit
qelderdelta@qelderdelta-UX331UA:~/Study/OS/lab6/build$ ps
PID TTY          TIME CMD
12523 pts/0    00:00:00 bash
13290 pts/0    00:00:00 ps

\end{verbatim}
\newpage
\section{Выводы}

В ходе выполнения лабораторной работы я познакомился с очередью сообщений ZeroMQ, а конкретнее с сокетами, реализующими паттерн Request-Reply, а также получил опыт написания распределенных систем.
\end{document}